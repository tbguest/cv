%%%%%%%%%%%%%%%%%%%%%%%%%%%%%%%%%%%%%%%%%
% Medium Length Professional CV
% LaTeX Template
% Version 2.0 (8/5/13)
%
% This template has been downloaded from:
% http://www.LaTeXTemplates.com
%
% Original author:
% Trey Hunner (http://www.treyhunner.com/)
%
% Important note:
% This template requires the resume.cls file to be in the same directory as the
% .tex file. The resume.cls file provides the resume style used for structuring the
% document.
%
%%%%%%%%%%%%%%%%%%%%%%%%%%%%%%%%%%%%%%%%%

%----------------------------------------------------------------------------------------
%	PACKAGES AND OTHER DOCUMENT CONFIGURATIONS
%----------------------------------------------------------------------------------------

\documentclass{resume} % Use the custom resume.cls style

\usepackage[left=0.75in,top=0.6in,right=0.75in,bottom=0.6in]{geometry} % Document margins

\name{Tristan Guest} % Your name

%\address{2078 Highway 329 \\ The Lodge, NS, B0J 1T0} % Your address
%%\address{123 Pleasant Lane \\ City, State 12345} % Your secondary addess (optional)
%\address{+1~(902)~830~0959 \\ tristanguest@gmail.com} % Your phone number and email

% \address{1355 Oxford Street \\ Halifax, NS, Canada, B3H 4R2} % Your address
%\address{123 Pleasant Lane \\ City, State 12345} % Your secondary addess (optional)
\address{+1~(902)~830~0959 \\ tristanguest@gmail.com} % Your phone number and email
\address{researchgate.net/profile/Tristan-Guest \\ github.com/tbguest} % Your phone number and email

% \begin{center}
%     \textbf{\Huge \scshape Tristan Guest} \\ \vspace{1pt}
%     \small 902-830-0959 $|$ \href{mailto:x@x.com}{\underline{tristanguest@gmail.com}} \\
%     \href{https://linkedin.com/in/...}{\underline{linkedin.com/in/tristan-guest}} $|$
%     \href{https://github.com/...}{\underline{github.com/tbguest}} $|$
%     \href{https://reseachgate.net/...}{\underline{researchgate.net/profile/Tristan-Guest}}
% \end{center}

\begin{document}

%----------------------------------------------------------------------------------------
%	EDUCATION SECTION
%----------------------------------------------------------------------------------------

\begin{rSection}{Education}

%{\bf University of California, Berkeley} \hfill {\em June 2004} \\ 
%B.S. in Computer Science \& Engineering \\
%Minor in Linguistics \smallskip \\
%Member of Eta Kappa Nu \\
%Member of Upsilon Pi Epsilon \\
%Overall GPA: 5.678

\textbf{Dalhousie University}, Halifax, Nova Scotia. \hfill Spring 2020 \\
Ph.D.$^{1}$, Physical Oceanography. \\
Dissertation: \textit{Morpho-sedimentary dynamics of a megatidal, mixed sand-gravel beach} \\
\begin{footnotesize}
	$^{1}$Transferred from M.Sc. program in 2017.
\end{footnotesize}


\textbf{Dalhousie University}, Halifax, Nova Scotia. \\
B.Sc. (combined honours), Mathematics and Oceanography. \hfill Fall 2013 \\

\end{rSection}

%----------------------------------------------------------------------------------------
%	WORK EXPERIENCE SECTION
%----------------------------------------------------------------------------------------


\begin{rSection}{Research Experience}
	
	\begin{rSubsection}{Advocate Beach, Bay of Fundy}{Fall 2018}{Scientist}{}
		\item Mixed sand-gravel beach field campaign. Measured the coevolution of beach surface elevation and sediment mean grain size in the intertidal zone using precision GPS, camera imagery, and acoustic range sensors.
	\end{rSubsection}
		
	\begin{rSubsection}{Aquatron pool tank, Dalhousie University}{Spring 2018}{Research Assistant}{}
		\item Laboratory experiment. Used ceiling-mounted cameras and fluorescent tracer floats (particle tracking velocimetry) to characterize the velocity dynamics of a partially constrained turbulent jet at the air-water interface.
	\end{rSubsection}
	
	\begin{rSubsection}{Atlantic Zone Monitoring Program (AZMP) Spring Cruise}{Spring 2016}{Research Assistant}{}
		\item Sea time aboard CCGS Hudson. Obtained and managed water samples for dissolved gas analysis.
	\end{rSubsection}
	
	\begin{rSubsection}{Advocate Beach, Bay of Fundy}{Fall 2015}{Scientist}{}
		\item Mixed sand-gravel beach field campaign. Measured steep beach hydro- and morphodynamic processes using buried pressure sensors, video, and GPS.
	\end{rSubsection}
	
	\begin{rSubsection}{Grand Passage, Bay of Fundy}{Summer 2013}{Research Assistant}{}
		\item Field experiment. Carried out a geotechnical characterization of the seafloor and profiled turbulence in Grand Passage.
	\end{rSubsection}

	\begin{rSubsection}{Advocate Beach, Bay of Fundy}{Fall 2013}{Scientist}{}
		\item Mixed sand-gravel beach field campaign. Measured swash and surf zone hydrodynamics using pressure sensors and video, alongside a sediment transport measurement program.
	\end{rSubsection}
	
	\begin{rSubsection}{Advocate Beach, Bay of Fundy}{Spring 2012}{Research Assistant}{}
		\item Field experiment. Observed nearshore processes and sediment transport using acoustic methods.
	\end{rSubsection}

\end{rSection}


\begin{rSection}{Peer-reviewed publications}
	
	{\em In preparation:}
	
	Guest, T. B. and A. E. Hay, Morpho-sedimentary dynamics of a megatidal, mixed sand-gravel beach.
	
	{\em Published:}
	
	Guest, T. B. and A. E. Hay (2021), Small-scale morpho-sedimentary dynamics in the swash zone of a megatidal mixed sand-gravel beach. \textit{Journal of Marine Science and Engineering}, 9(4):413
	
	Guest, T. B. and A. E. Hay (2019), Timescales of beach cusp evolution on a steep, megatidal, mixed sand-gravel beach. \textit{Marine Geology}, 416, 105984.
	
	Guest, T. B. and A. E. Hay (2017), Vertical structure of pore pressure under surface gravity waves on a steep, megatidal, mixed sand-gravel-cobble beach. \textit{Journal of Geophysical Research: Oceans}, 122, 153–170.
	
\end{rSection}


\begin{rSection}{Conference and Institutional Talks}
	
	Guest, T. B. Investigating the role of grain size in beach morphological change: Insights from a megatidal mixed sand-gravel beach. \textit{Dalhousie Oceanography Departmental Seminar Series}, 12 May 2020, Halifax, NS, Canada.
	
	Guest, T. B. and A. E. Hay. Swash zone morpho-sedimentary dynamics on a megatidal, mixed sand-gravel beach. \textit{11th River, Coastal, and Estuarine Morphodynamics Symposium (RCEM)}, 20 November 2019, Auckland, NZ.
	
	Guest, T. B. and A. E. Hay. Cobble dynamics on a mixed sediment substrate. \textit{Conference for Dalhousie Oceanography Graduate Students (CDOGS)}, 22 March 2019, Halifax, NS, Canada.
	
	Guest, T. B. and A. E. Hay. Timescales of beach cusp evolution on a megatidal, mixed sand-gravel beach. \textit{Conference for Dalhousie Oceanography Graduate Students (CDOGS)}, 23 March 2018, Halifax, NS, Canada.
	
	Guest, T. B. and A. E. Hay. Timescales of beach cusp evolution on a megatidal, mixed sand-gravel beach. \textit{Ocean Sciences Meeting (OSM)}, 14 February 2018, Portland, OR, USA.
	
	Guest, T. B. and A. E. Hay. Vertical structure of pore pressure under surface gravity waves on a steep, megatidal, mixed sand-gravel-cobble beach. \textit{American Geophysical Union (AGU) Fall Meeting}, 14 December 2016, San Francisco, CA, USA.
	
	Guest, T. B. and A. E. Hay. Vertical structure of pore pressure under surface gravity waves on a steep, megatidal, mixed sand-gravel-cobble beach. \textit{Bedford Institute of Oceanography (BIO) Ocean and Ecosystem Science Seminar Series}, 14 October 2016, Dartmouth, NS, Canada.
	
	Guest, T. B. and A. E. Hay. Pressure response of a sand and gravel bed to water waves. \textit{Conference for Dalhousie Oceanography Graduate Students (CDOGS)}, 18 March 2016, Halifax, NS, Canada.
	
	Guest, T. B. and A. E. Hay. Mixed sediment beaches: Cusps and edge waves. \textit{Conference for Dalhousie Oceanography Graduate Students (CDOGS)}, 20 March 2015, Halifax, NS, Canada.
	
\end{rSection}


\begin{rSection}{Institutional Activities}
	
	\begin{rSubsection}{Current Tides Magazine}{2018-2020}{Editor-in-Chief}{}
		\item Oversaw the production and launch of Volume 4 of the Dalhousie Oceanography student research magazine (http://currenttides.ocean.dal.ca/)
		\item Responsible for obtaining funding, managing a team of 11 authors and 7 editors, graphic design, orchestrating layout and print production processes, and hosting a launch event
	\end{rSubsection}
	
	\begin{rSubsection}{Current Tides Magazine}{2017-2018}{Assistant Editor}{}
		\item Provided editorial assistance in the production of Volume 3 of the Dalhousie Oceanography student research magazine
	\end{rSubsection}
	
	\begin{rSubsection}{Oceanography 1000: Conversations with Ocean Scientists}{2017-2018}{Teaching Assistant}{}
		\item First year science writing course
		\item Guided students through writing of original academic articles in weekly tutorials
	\end{rSubsection}
	
\end{rSection}
%----------------------------------------------------------------------------------------
%	TECHNICAL STRENGTHS SECTION
%----------------------------------------------------------------------------------------


%--------PROGRAMMING SKILLS------------
%\section{Programming Skills}
%  \resumeSubHeadingListStart
%    \item{
%      \textbf{Languages}{: Scala, Python, Javascript, C++, SQL, Java}
%      \hfill
%      \textbf{Technologies}{: AWS, Play, React, Kafka, GCE}
%    }
%  \resumeSubHeadingListEnd

\begin{rSection}{Technical Skills}
	
	\begin{tabular}{ @{} >{\bfseries}l @{\hspace{6ex}} l }
		Languages \& Software & \textit{Proficient in}: Python, JavaScript/Node.js, HTML/CSS, MATLAB, Git. \\
		~ & \textit{Experience with}: Bash, LaTeX, SQL, \\
		~ & \hspace{80px} Ruby, R, C, Fortran.\\
%		~ & R, SQL \\
		Operating Systems & Windows, Unix \\
		Grahics & Inkscape, Gimp, Sketchup \\
		Areas of Interest & Image processing/computer vision \\
		~ & Linux network administration \\
		~ & Microcontrollers \& sensor interfacing \\
		~ & Real time kinematic (RTK) GPS surveying \\
		~ & Front end web development \\
		
	\end{tabular}
	
\end{rSection}

%\begin{rSection}{Technical Strengths}
%
%Linux network administration
%\item Command line programming and shell scripting
%\item Remote access
%\item Time synchronisation (NTP/PTP)
%\item Static/dynamic addressing\\
%
%Image processing
%- segmentation/thresholding
%- edge detection
%- feature tracking
%- object identification
%- Particle image/tracking velocimetry (PIV/PTV)
%- Rectification
%
%Microcontrollers and embedded electronics
%- Sensing systems (e.g. pressure, temperature, range, optical imagery, wind speed)
%- Data logging
%- self-contained 12 V systems
%
%RTK GPS
%
%\end{rSection}



% \begin{rSection}{Personal Pursuits}

% 	Woodworking (boatbuilding \& restoration, furniture, housewares, structures)
	
% 	Boating (oar, paddle, \& sail)
	
% 	Cycling (road \& mountain)
	
% 	Board sports (skate \& snow)	
	
% \end{rSection} 

% awards? (NSGS)
% Current Tides

%----------------------------------------------------------------------------------------
%	EXAMPLE SECTION
%----------------------------------------------------------------------------------------

%\begin{rSection}{Section Name}

%Section content\ldots

%\end{rSection}

%----------------------------------------------------------------------------------------

\end{document}
